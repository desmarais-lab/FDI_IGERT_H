\documentclass{article}
\usepackage[margin=1in]{geometry}
\usepackage{longtable}
\usepackage{hyperref}
\hypersetup{
    colorlinks=true,
    linkcolor=red,
    filecolor=magenta,      
    urlcolor=red,
}
\usepackage{graphicx}
\usepackage{setspace}
\usepackage[table]{xcolor}
\usepackage{booktabs}
\usepackage{courier}
\begin{document}

\title{The Network of Foreign Direct Investment Flows: \\Theory and Empirical Analysis}
\author{John  Schoeneman\thanks{\footnotesize{
jbs5686@psu.edu, PhD Student, Pennsylvania State University.}} \and Boliang Zhu\thanks{\footnotesize{bxz14@psu.edu, Assistant Professor, Department of Political Science, Pennsylvania State University. }} \and Bruce A. Desmarais\thanks{\footnotesize{
bdesmarais@psu.edu, Associate Professor, Department of Political Science, Pennsylvania State University.}}}
\date{}
\maketitle

\singlespacing
\begin{abstract} 
    \noindent We study the structure of the international network of foreign direct investment (FDI) flows. The political economy of FDI literature has established several theoretical claims and empirical regularities regarding exogenous political and economic determinants of FDI inflows. These include security alliances, preferential trade agreements, migration networks, and colonial history. However, existing studies---based on monadic and to a lesser degree, dyadic regression models---overlook the complex dependencies that are likely to characterize the network. Recent developments in methodology for studying international relations show that the regression framework is typically inadequate for quantitatively modeling dyadic relational data, such as FDI flows. In this paper, we integrate hypotheses regarding exogenous determinants and novel hypotheses regarding structural dependencies into a comprehensive exponential random graph model (ERGM) for weighted networks. Our findings reveal that the FDI flow network  exhibits a number of complex dependencies, such as reciprocity, that have been omitted from previous empirical models of FDI flows.

\end{abstract}

\section{Introduction}

Research examining foreign direct investment (FDI) and its relationship with economic and political determinants is expansive. Much of this work is conducted using the gravity model, which was originally developed to predict trade flows. This framework models FDI flows using dyadic data and the product of partner GDPs as mass and some variant of distance as an independent variable. Our work highlights a key weakness of these models that rely on standard panel regression models. There has been a growing body of literature that brings into question the way we estimate models for dyadic data (add list of papers). The primary challenge is that dyadic data is an edge-list and therefore represents a network. Ignoring this unmodeled network structure violates assumptions within a generalized linear model, potentially leading to biased estimates.


\end{document}