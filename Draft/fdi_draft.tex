\documentclass{article}
\usepackage[margin=1in]{geometry}
\usepackage{longtable}
\usepackage{hyperref}
\hypersetup{
    colorlinks=true,
    linkcolor=black,
    filecolor=magenta,      
    urlcolor=red,
    citecolor=cyan
}
\usepackage{graphicx}
\usepackage{setspace}
\usepackage[table]{xcolor}
\usepackage{booktabs}
\usepackage{courier}
\begin{document}

\title{The Network of Foreign Direct Investment Flows: \\Theory and Empirical Analysis}
\author{John  Schoeneman\thanks{\footnotesize{
jbs5686@psu.edu, PhD Student, Pennsylvania State University.}} \and Boliang Zhu\thanks{\footnotesize{bxz14@psu.edu, Assistant Professor, Department of Political Science, Pennsylvania State University. }} \and Bruce A. Desmarais\thanks{\footnotesize{
bdesmarais@psu.edu, Associate Professor, Department of Political Science, Pennsylvania State University.}}}
\date{}
\maketitle

\singlespacing
\begin{abstract} 
    \noindent We study the structure of the international network of foreign direct investment (FDI) flows. The political economy of FDI literature has established several theoretical claims and empirical regularities regarding exogenous political and economic determinants of FDI inflows. These include security alliances, preferential trade agreements, migration networks, and colonial history. However, existing studies---based on monadic and to a lesser degree, dyadic regression models---overlook the complex dependencies that are likely to characterize the network. Recent developments in methodology for studying international relations show that the regression framework is typically inadequate for quantitatively modeling dyadic relational data, such as FDI flows. In this paper, we integrate hypotheses regarding exogenous determinants and novel hypotheses regarding structural dependencies into a comprehensive exponential random graph model (ERGM) for weighted networks. Our findings reveal that the FDI flow network  exhibits a number of complex dependencies, such as reciprocity, that have been omitted from previous empirical models of FDI flows.

\end{abstract}

\section{Introduction}

Research examining foreign direct investment (FDI) and its relationship with economic and political determinants is expansive. Much of this work is conducted using the gravity model, which was originally developed to predict trade flows. This framework models FDI flows using dyadic data and the product of partner GDPs as mass and some variant of distance as an independent variable. Our work highlights a key weakness of these models that rely on standard panel regression models. There has been a growing body of literature that brings into question the way we estimate models for dyadic data (add list of papers). The primary challenge is that dyadic data is an edge-list and therefore represents a network. Ignoring this unmodeled network structure violates assumptions within a generalized linear model, potentially leading to biased estimates.

\section{Methods Background}

\section{Theory}

\section{Data and Research Design}

\subsection{FDI inflows}

For the dependent variable we are using bilateral FDI inflows. These are from the United Nations Conference on Trade and Development (UNCTAD) and were first made available in 2014 \cite{UNCTAD}. We use the entire time-period available, which is 2001-2012. Past work that has looked at country-year FDI relationships relied on monadic data. The advantage of using dyadic data is that it not only lets us model network relationships, but the disaggregation allows us to measure changes in FDI inflows related to covariates that are at the dyad level, such as PTAs. Because FDI flows tend to be highly dispersed we use the natural log of the values. 

\subsection{Network Statistics}

\subsection{Covariates}

We control for both economic and political variables. Following the literature's standard for predicting FDI inflows we include standard gravity variables. This includes the log product of the dyad's GDP and logged euclidean distance. Generally, larger products of GDP are associated with higher levels of FDI while longer distances are associated with less FDI \cite{MZ2011, WB1}. \\

The other key economic variable that is included trade in intermediate goods\cite{OECD}. This is constructed according to the UN Broad Economic Category classification definitions that separates goods by end-use category. Intermediate goods include unprocessed and partially processed agricultural goods and industrial goods. Past research has shown that FDI and trade are compliments \cite{AN} and the advantage of using trade in intermediate goods is that it proxies for production supply chains, which we expect to be strongly positively associated with FDI inflows.\\

A more minor economic variable included is the growth rate of the economy, which has been used in past studies to stand in for the general health of a country's economy \cite{WB2}.\\

We include two categories of international agreement variables. The first is dummy variables for defense agreements \cite{Gibler09}. This includes defense, entente, non-aggression, and neutrality treaties. We expect these variables to positively associated with FDI inflows, particularly defense treaties since this indicates political cooperation and a lower risk of expropriation. The second international agreement variable is preferential trade agreement (PTA) depth \cite{dur2014design}. Past work has argued that PTAs represent a commitment to liberal markets that investors would favor and therefore would be associated with increased FDI inflows \cite{BM14}. However, this study used monadic data and only a count PTAs signed. Our work takes this further since we are using network data and increases in PTA depth and FDI inflows is measured at the dyadic level and the PTA variable we use is built using latent trait analysis with 48 different variables.\\

There is substantial amount of work that explores the relationship between regime type and FDI inflows. The work is inconclusive, but include controls for it nonetheless using the Polity IV score \cite{polity2012polity}. We also include political violence to proxy for state stability, which expect to be negatively correlated with FDI inflows \cite{marshall2005major}.

\subsection{Covariate Summary Statistics}

\section{Results}

\section{Conclusion}


\newpage
\bibliographystyle{plain}
\bibliography{fdi_reference}


\end{document}