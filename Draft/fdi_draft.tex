\documentclass[reqno,onecolumn,letterpaper,12pt]{article}
\usepackage[margin=1in]{geometry}
\usepackage{longtable}
\usepackage{bm}
\usepackage{hyperref}
\hypersetup{
    colorlinks=true,
    linkcolor=black,
    filecolor=magenta,
    urlcolor=red,
    citecolor=cyan
}
\usepackage{graphicx}
\usepackage{rotating}
\usepackage{amsmath}
\usepackage{setspace}
\usepackage{amssymb}
\usepackage{natbib}
\usepackage{amsfonts}
\usepackage[table]{xcolor}
\usepackage{booktabs}
\usepackage{subcaption}
\usepackage{courier}

\newcommand\citeapos[1]{\citeauthor{#1}'s (\citeyear{#1})}
\newcommand{\R}{\texttt{R}} % Write R in typewriter font

\begin{document}

\title{The Network of Foreign Direct Investment Flows: \\Theory and Empirical Analysis\footnote{This work was supported by NSF grants SES-1558661, SES-1619644, SES-1637089, CISE-1320219, SMA-1360104, and IGERT Grant DGE-1144860. Any opinions, findings, and conclusions or 61 recommendations are those of the authors and do not necessarily 62 reflect those of the sponsor.}}
\author{John  Schoeneman\thanks{\footnotesize{
jbs5686@psu.edu, PhD Student, Pennsylvania State University.}} \and Boliang Zhu\thanks{\footnotesize{bxz14@psu.edu, Assistant Professor, Department of Political Science, Pennsylvania State University. }} \and Bruce A. Desmarais\thanks{\footnotesize{
bdesmarais@psu.edu, Associate Professor, Department of Political Science, Pennsylvania State University.}}}
\date{}
\maketitle

\singlespacing
\begin{abstract}
    \noindent We study the structure of the international network of foreign direct investment (FDI). The political economy of FDI literature has established several theoretical claims and empirical regularities regarding determinants of FDI. However, existing studies---based on regression models---overlook the complex dependencies that are likely to characterize the FDI network. Recent developments in methodology for studying international relations show that regression is inadequate for quantitatively modeling dyadic data. We integrate hypotheses regarding exogenous determinants and novel hypotheses regarding structural network dependencies into an exponential random graph model (ERGM) for weighted networks. Our findings reveal that the FDI network exhibits both reciprocity and transitivity in the FDI network. These dependencies have been omitted from previous empirical models of FDI flows, which has consequences for inferences regarding covariate effects. In addition to our substantive findings, we offer a broad methodological contribution by introducing the ERGM for count-weighted networks in political science.

\end{abstract}
~\\

{\bf Pre-submission to-do}
\begin{itemize}
\item add descriptive statistics---bar charts giving proportion 1 by year for dichotomous variables, and yearly box-plots for quantitative variables. {\bf John, would you work on this?}
\item Revise interpretation plot so that they don't sound like actual residuals from the model {\bf BD}: modified precisely what this plot was depicting, and discussed it in terms of effects and predicted values.
\item Produce goodness of fit plots that represent network structure, not just BIC.
\item Change variables
\begin{itemize}
\item attempt to add any variables we don't have that are in Quan Li and Vashchilko Biilateral FDI, dyadic Military conflict, simplify treaty variables to match. Figure out what the weighting matrix is.
\end{itemize}
\item re-estimate combining years
\item Overview, list or summary of valued dyadic analysis in IPE.
\item Elaborate our dependence theories
\item rug plot or density plot on interpretations.
\item  revise intro and conclusion {\bf BD:} Revised to discuss methods contributions in the intro, conclusion, and abstract.
\end{itemize}

\clearpage
\doublespacing
\section{Introduction}


What accounts for the pattern of global foreign direct investment (FDI) flows? Standard economic models attribute cross-border capital movements primarily to relative factor endowments, market size, and transportation and trade cost \citep[see,~e.g.,][]{Helpman:1984,Carr_et_al:2001}. Yet, footloose capital becomes immobile ex post and thus an ``obsolescent bargain,'' which is vulnerable to host government's expropriation \citep{Vernon:1971,Vernon:1980}. Building on this insight, the recent political economy of FDI literature emphasizes the importance of political institutions in constraining host government's opportunistic behavior. Scholars suggest that political constraints \citep{Henisz:2000}, democratic governance \citep{Jensen:2003,Jensen:2006}, rule of law \citep{Li_Resnick:2003,Staats_Biglaiser:2012}, and participation in international institutions \citep{Buthe_Milner:2008,Allee_Peinhardt:2011} help to ensure policy credibility and provide investor protection, thereby luring in foreign investors.

To date, existing theories have focused exclusively on firm-level characteristics and home- and host-country economic and political parameters to explain cross-border FDI flows. One implicit assumption in these theories is that countries and dyads are independent of each other. This assumption, nonetheless, unlikely holds, given the intertwined links among multinational corporations (MNCs) and the expansion of global production networks \citep{UNCTAD:2013}. If global FDI flows can arise endogenously from the network structure, existing political economy models of FDI remain incomplete by excluding high-order structural variables. Furthermore, neglecting network structure variables may lead to biased estimates or even invalid inferences \citep{cranmer2011inferential}.

We argue that two network structures---reciprocity and transitivity---are important to account for the pattern of cross-border FDI flows. First, reciprocity arises from the fact that FDI represents an oligopolistic expansion strategy of MNCs and the fact that host governments tend to use a principle of reciprocity to regulate FDI inflows. Second, the expansion of global supply chains and the diffuse of preferential trade agreements (PTAs) drive the transitivity/clustering of investment activities. Utilizing bilateral FDI flow data from United Nations Conference on Trade and Development (UNCTAD) over the period of 2001--2012, we find strong evidence that FDI inflows are reciprocal and transitive/clustering, suggesting that cross-border FDI flows are interdependent and shaped by their network structure. We further show that ignoring high-order network structure variables can lead to biased estimates in standard panel regression models.

To test our arguments, we use the count exponential random graph model (ERGM) \citep{krivitsky2012exponential}. To our knowledge, this recent extension of ERGM has not yet been applied in political science research. As such, our use and introduction of the count ERGM represent two distinct contributions. First, the application of the count ERGM to the study of bilateral FDI results in novel findings regarding patterns of dependence that characterize the FDI network. Second, by introducing the count ERGM in political science, we we provide an illustrative application of a methodology that is widely applicable in political science research. The count ERGM can be applied to any network in which ties are count-weighted, and therefore represents a valuable tool for political scientists, who regularly study networks with count-weighted ties (e.g., interstate trade \citep{ward2007persistent}, shared membership in international governmental organizations \citep{boehmke2016addressing}, the count of bills co-sponsored between legislators \citep{kirkland2013hypothesis}, the number of policy ideas on which policymakers and other policy stakeholders agree \citep{leifeld2013reconceptualizing}).

We organize the paper as follows. In the next section we provide an overview of the assumption of independence in empirical research on FDI, and international relations in general. Following that, we present theoretical claims that the FDI network should be characterized by reciprocity and transitivity. Finally, we discuss the research design and present empirical results.



%Research examining foreign direct investment (FDI) and its relationship with economic and political determinants is expansive. Much of this work is conducted using the gravity model, which was originally developed to predict trade flows. This framework models FDI flows using dyadic data and the product of partner GDPs as mass and some variant of distance as an independent variable. Our work highlights a key weakness of these models that rely on standard panel regression models. There has been a growing body of literature that brings into question the way we estimate models for dyadic data. The primary challenge is that dyadic data is an edge-list and therefore represents a network. Ignoring this unmodeled network structure violates assumptions within a generalized linear model, particularly independence, potentially leading to biased estimates. While there is a growing body of work that has addressed the interdependence problem in dyadic data, especially for trade, FDI has been largely ignored. We address this gap in the literature through the use of network modeling to test previous theories alongside of estimating dependence terms.


%{\bf [John or Boliang, paragraph on what scholars have asked or know about FDI]}



\section{Independence Assumptions and the Study of FDI}

The dominant eclectic paradigm suggests that MNCs arise from taking advantage of firms' intangible or specific assets to overcome imperfections in arm's length transactions \citep{Caves:1996,Dunning:1992}. In this sense, direct investment or the establishment of a foreign affiliate is a decision made by a parent company. Yet, footloose capital becomes relatively immobile after investment takes place, and thus a hostage to the host government \citep{Vernon:1971,Vernon:1980}. MNCs are ex post vulnerable to the host government's opportunistic behavior, such as asset expropriation or subtle policy changes, which dampens firms' profitability. Adopting a neo-institutionalist approach, the political economy of FDI literature emphasize the role of domestic and international institutions in preventing state's predatory behavior and ensuring credible commitment, thereby attracting FDI \citep[e.g.][]{Henisz:2000,Jensen:2003,Jensen:2006,Li_Resnick:2003,Staats_Biglaiser:2012,Buthe_Milner:2008,Allee_Peinhardt:2011}.

There is now a large empirical literature examining the determinants of FDI inflows \citep[e.g.,][]{Noorbakhsh_et_al:2001,Yeaple:2003,Jensen:2003,Li_Resnick:2003,Buthe_Milner:2008,Li_Vashchilko:2010,Kerner:2009,Wright_Zhu:2017}.\footnote{See \citet{Pandya:2016} for a comprehensive review.} Existing studies typically model FDI flows at the monadic and to a lesser extent at the dyadic level. One implicit assumption in existing theoretical and empirical models is that FDI flows into one country or between one dyad are independent of other countries or dyads. As such, the political economy of FDI literature has focused on examining the impact of host countries' political and   economic characteristics on FDI flows, while overlooking the structure parameters. Given the intertwined linkages among MNCs and the expansion of global production networks \citep{UNCTAD:2013}, we expect that high-order network structures should play an important role in shaping the pattern of FDI flows.

% Paragraph on interdependence in IR/IPE
The study of FDI is not unique in its reliance on the independence assumption. Historically, statistical models used in international relations have involved the implicit assumption that countries and dyads are independent of each other \citep{diehl2016conditional,ward2007persistent}. In the conflict literature, hypotheses regarding the likelihood of interstate war are often tested using dyadic panel data without a close attention to high-order structural dependencies \citep[e.g.][]{Cranmer_Desmarais:2011}.\footnote{\citet{ward2007disputes} and \citet{ward2007persistent} are notable exceptions.} Likewise, in the IPE literature a dyadic panel research design is a common practice in the studies of bilateral flows of trade \citep[e.g.][]{Mansfield_et_al:2000,Rose:2004,Goldstein_et_al:2007,Bliss_Russett:1998,Gowa_Mansfield:1993}, capital \citep[e.g.][]{Li_Vashchilko:2010,Leblang:2010,Egger_Pfaffermayr:2004}, aid \citep[e.g.][]{BDM_Smith:2009}, and migrants \citep[e.g.][]{Fitzgerald_et_al:2014}. For instance, the well-known democratic trade hypothesis, i.e. whether democratic dyads trade more with each other than other types of dyads, is examined using dyadic panel data sets where pairs of nations are assumed to be independently distributed. \footnote{\citet{Erikson_et_al:2014} point out that a dyadic panel research design results in overconfident significance tests in OLS regressions if observations are not independent. They propose randomization testing to adjust standard errors. In this paper, we directly model the structural characteristics of the examined dependent variable. }


This assumption is now widely viewed as dubious \citep[see, e.g., ][]{ward2007persistent, chu2010homogenization,cranmer2016critique,dorff2013networks,lee2013network,howell2013geography,kinne2016agreeing}. The negative consequences of erroneously assuming independence are two-fold. First, the model is misspecified, which leads to biased estimates and hypothesis tests for covariates included in the model. Second, researchers arrive at a limited theoretical scope in which they only consider the relationship between the dependent variable and covariates, and do not consider the influences that ties and countries have on each other. The methodological toolkit available to scholars of international relations has advanced well beyond conventional regression approaches, and now offers at least three prominent options for modeling interdependence in relational data----stochastic actor oriented models \citep[e.g., ][]{camber2010geometry,kinne2016agreeing,kinne2013network,kinne2014dependent,warren2016modeling}, exponential random graph models \citep[e.g.,][]{cranmer2012complex,cranmer2012toward,raeymaeckers2016influence}, and latent space models \citep[e.g., ][]{ward2007disputes,ward2013gravity,metternich2013antigovernment}. As such, it is quite methodologically feasible to move beyond questionable independence assumptions in the study of FDI.


\section{Dependence Hypotheses in FDI Flows}

The primary theoretical advantage of taking a network approach to studying FDI is that we can develop and test hypotheses regarding a novel class of effects---the effects that tie in the FDI network have on each other. In deriving our theoretical claims regarding network dependence, we focus on the operating characteristics of MNCs, global production networks (GPNs), and preferential trade agreements (PTAs), which are central to the FDI process. Through consideration of the structure and function of MNCs, we derive a reciprocity \citep{garlaschelli2004patterns} hypothesis---a claim that, all else equal, investments from state $i$ will flow disproportionately to state $j$ if firms from state $j$ hold a high stock of investments in state $i$. Through consideration of GPNs and PTAs, we derive a hypothesis of transitivity \citep{holland1971transitivity}---that investments from firms in state $i$ will flow disproportionately to state $j$ to the degree that there are third-party states $k$ in which states $i$ and $j$ both exchange high investment flows.

\subsection{Reciprocity of FDI Flows}
It is well known that international trade is conducted based on the principle of reciprocity under the GATT/WTO regime in the sense that governments lower tariffs reciprocally to neutralize the terms-of-trade externality \citep{Bagwell_Staiger:1999}. Yet, reciprocity embedded in traditional bilateral investment treaties (BITs) concerns more the equal treatment and protection of investors, not liberalization or exchanges of market opportunities \citep[~56]{DiMascio_Pauwelyn:2008}. The reciprocity of FDI, instead, stems from the fact that FDI represents an oligopolistic expansion strategy of MNCs \citep{Hymer:1976,Kindleberger:1969}. MNCs arise from exploiting their ownership-specific assets to overcome imperfections in arm's-length markets \citep{Caves:1996,Dunning:1992}. These proprietary assets include, for examples, advanced technology, brand names, product differentiation, and managerial and advertising skills, which are of a public-goods character and possess substantial economies of scale. To make the most use of these firm-specific assets and best exploit economies of scale, MNCs actively seek to expand market shares and penetrate each other's home markets with highly differentiated products, resulting in reciprocal flows of investment.\footnote{The reciprocal investment flows can also result from firms' rivalistic strategy in response to foreign entries. Foreign entry may generate disruptive effects in the market, which stimulates rivalrous expansion of local firms into the home market of the foreign firm when two conditions are met: 1) local firms possess intangible assets that enable them to exploit rents in the foreign market; 2) their entry could disrupt the home market of the foreign firm \citep{Graham:1978}. }

Historically, global investment activities have been dominated by MNCs from developed countries and characterized by a pattern of two-way flows.\footnote{Over the past decade, we have witnessed a surge of direct investment from emerging-market MNCs. Meanwhile, developing countries become increasingly popular investment destinations. In 2012, developing countries as a whole received more FDI than developed countries for the first time ever \citep{UNCTAD:2013}.} FDI mainly flows between pairs of developed countries countries, even in the same industries, most of which investments are horizontal and market-seeking \citep[171]{Markusen:1995}. \citet[~22]{Julius:1990} reports that during the 1980s the percentage of FDI circulating within France, Japan, West Germany, United Kingdom, and United States (G-5) had risen to 75\%. Even in 2010, the figure remained high at 53\%; among G-7 countries including Canada and Italy, 65\% of G-7 outward FDI was absorbed by other G-7 countries.\footnote{Authors' calculations based on the UCTAD bilateral FDI statistics.}

MNCs' oligopolistic expansion often encounters opposition from both host governments and the public due to concerns of national security, market monopoly, and protection of indigenous firms. In order to gain access to foreign markets, MNCs have incentives to leverage their influence on home governments to exchange market accesses with foreign governments, thereby establishing or reinforcing reciprocity \citep{Milner:1988,Crystal:2003}. As \citet[6]{Crystal:2003} note, ``they [MNCs] want to counter the existing restrictions---on both trade and FDI---that some foreign countries have imposed and so therefore will favor contingently restrictive policies.'' Indeed, both governments and citizens' reactions to inward foreign investment follow a priciple of reciprocity. \citet{Tingley:2015} show that U.S. government officials are more likely to oppose Chinese firms' mergers and acquisitions when China has blocked U.S. investment. Recently, the India government is proposing a reciprocity-based policy towards foreign investment. Piyush Goyal, the Minister of State Power, Coal, Renewable Energy, and Mines, said in an interview, ``India won't allow power companies to invest from countries where Indian firms are banned.''\footnote{Singh, Sarita. ``India to Give a `Power' Blow to Chinese Firms Soon.'' \textit{The Economic Times}, May 22, 2017. \url{http://economictimes.indiatimes.com/industry/energy/power/security-concerns-indias-new-rules-to-bar-chinese-companies-in-power-sector/articleshow/58780085.cms}, Accessed June 6, 2017.} Among the public, recent experimental evidence shows that citizens are more likely to support foreign investment from countries that grant reciprocal market access \citep{Chilton_et_al:forthcoming}. Therefore, we hypothesize the following:



\begin{center}
\textit{Hypothesis 1: FDI flows are reciprocal.}
\end{center}

%\textit{Hypothesis 1a: The reciprocity of FDI flows is stronger among developed dyads than others.}

\subsection{Transitivity/Clustering of FDI Flows}
Two factors are likely to drive the transitivity/clustering of investment activities---the expansion of global production networks and the diffusion of PTAs.  One distinct feature of today's globalization is the increasing fragmentation of production processes and the dramatic expansion of global supply chains \citep{UNCTAD:2013}. At the center of global production networks are MNCs, which coordinate global supply chains through complex networks of their foreign affiliates, subcontractors, or arm's-length suppliers \citep[xxii]{UNCTAD:2013}. These intertwined networks give rise to the clustering of FDI activities. In a most straightforward way, MNCs' establishment of a foreign affiliate is typically followed by investment of their partners, such as upstream suppliers or downstream purchasers, who themselves are often multinationals that coordinate their own networks of supply chains. These types of interdependent linkages lead to multiple triangle closures of investment flows. Consider a case of three countries: A, B, and C. Suppose firms from A invest in B as suppliers to firms in B.\footnote{Alternatively, firms in A can export intermediate goods to B. However, firms typically favor near suppliers. Moreover, if transportation and trade costs between A and B are high, firms in A will prefer direct investment over export \citep{Carr_et_al:2001}. } If firms in B establish foreign affiliates in C to exploit locational advantages such as a large consumer market or favorable government policies, investment by their suppliers from A likely follows to serve these foreign affiliates. For instance, Volkswagen's investment in Skoda Auto in Czech Republic not only attracted other auto makers such as PSA Peugeot and Toyota, but also international suppliers of parts and components to acquire local firms or build new factories; ``As of 2002, there were 270 firms operating in the Czech Republic, representing 45 percent of the top 100 world suppliers of automotive parts and components.'' \citep[352]{Kaminski_Javorcik:2005}. Likewise, Volkswagen's recent investment in Ningbo-Hangzhou Bay New Zone in China has brought in suppliers from South Korea, France, and the United States.\footnote{\url{http://cepz.ningbo.gov.cn/cat/cat159/con_159_5310.html}, Accessed June 7, 2017.}

More importantly, global supply chains tie countries together and significantly increase the cost of governments' opportunistic behaviors---such as expropriation or subtle policy changes. Political risk in host countries remains a primary concern of investors since footloose capital becomes an ``obsolescent bargain'' due to its ex post immobility \citep{Vernon:1971,Vernon:1980}. Global production networks significantly constraint governments' policy discretion, because the proper functioning of the supply chains hinges crucially on the cooperation and coordination of the countries involved. For example, even Starbucks, a company that has a relatively simple supply chain, ``sources coffee from thousands of traders, agents and contract farmers across the developing world; manufactures coffee in over 30 plants, ...; distributes the coffee to retail outlets through over 50 major central and regional warehouses and distribution centres; and operates some 17,000 retail stores in over 50 countries across the globe'' \citep[142]{UNCTAD:2013}.

Apparently, any interruption in the global supply chain can severely damage Starbucks's business. Thus, governments are incentivized to refrain from arbitrary interventions or even subtle policy changes that dampen firms' profitability levels. Especially when two countries are integrated into the same global production network coordinated by leading MNCs in a third country, the risk-mitigating effect of the network is magnified. This is because all countries involved have strong incentives to ensure that the network functions well. \citet{johns2016under} show that host governments are less likely to expropriate foreign firms when they are closely connected to firms in host countries through supply chains. \citet{Kim_Solingen:2017} find that East Asian countries that are deeply integrated into global production networks are more likely to promote cooperation and peace between each other. Therefore, we expect that FDI has a high probability to flow among countries that are in the same global production network, resulting in the transitivity/clustering of investment activities.\footnote{In a broader term, when two countries are tightly linked to a third country through investment flows, FDI should be more likely to flow between these two countries due to shared economic interests and reduced political risk. }

The diffusion of PTAs is likely to drive the clustering of direct investment activities as well. The formation of a PTA eliminates trade barriers among member states. The removal of trade barriers allows MNCs to optimize their global supply chains and fragment its production stages within member states to best capitalize on locational advantages such as factor endowments and favorable government policies. For instance, with the increasing integration of the European Community, the 1980s witnessed a restructuring of many industries and regionalization of MNC activities to exploit the advantages of a single market, leading to a surge of intra-region FDI \citep[34]{UNCTAD:1991}. Importantly, most favored-nation treatment, investment clauses, and dispute-settle mechanisms that are embedded in PTAs help to alleviate foreign investors' concerns of government interventions, discrimination, and expropriation \citep{Buthe_Milner:2008,buthe2014foreign}, thereby making member states more attractive investment destinations to each other. PTAs therefore reinforce the transitive clustering of investment activities.

\begin{center}
\textit{Hypothesis 2: FDI flows are likely to be transitive.}
\end{center}




\section{Data and Research Design}


To test our hypotheses, we estimate a gravity model of FDI flows as follows:

% add the regression equation

\begin{equation}\label{gravity}
  logFDI_{ijt}=\alpha + \beta_{1}*GDP~Product + \beta_{2}*GDP/cap_{it} + \beta_{3}*GDP/cap_{jt} +
\end{equation}

The dependent variable is bilateral FDI inflows. The data are from UNCTAD, covering the time-period of 2001 to 2012. The data set was first made available in 2014 \citep{UNCTAD:2014}. Most existing empirical studies on FDI use monadic data because scholars are primarily interested in how host countries' economic and political characteristics affect capital inflows.\footnote{There are very few studies that use dyadic FDI data. See \citet{Frenkel_et_al:2004}, \citet{Leblang:2010}, \citet{Li_Vashchilko:2010}, and \citet{Razin_et_al:2005}. } The advantage of using dyadic data is that it allows us not only to model network relationships, but to measure changes in FDI inflows related to covariates that are at the dyad level, such as PTAs, alliances, colonial history, and common language. We take the natural log of the bilateral FDI flow variable to deal with the skewed distribution.

\subsection{Covariates}

\begin{figure}[htp]
\centering
\begin{tabular}{c@{\hskip -.4cm}c}
Alliance Treaty &
Defense Treaty\\
\includegraphics[height=.22\textheight, clip=true, trim=0cm 1cm 0cm 1.6cm]{draft_figures/descriptive_plots/alliance.pdf}    &
\includegraphics[height=.22\textheight, clip=true, trim=.5cm 1cm 0cm 1.6cm]{draft_figures/descriptive_plots/defense.pdf}   \\
\end{tabular}
\caption{\label{fig:effectPlots4} Summary Statistics.}
\end{figure}

\begin{figure}[htp]
\centering
\begin{tabular}{c@{\hskip -.4cm}c}
Bilateral Investment Treaty &
Polity\\
\includegraphics[height=.22\textheight, clip=true, trim=0cm 1cm 0cm 1.6cm]{draft_figures/descriptive_plots/BIT.pdf}    &
\includegraphics[height=.22\textheight, clip=true, trim =1cm 1cm 0cm 1.6cm]{draft_figures/descriptive_plots/Polity.pdf}   \\
Dyad GDP Product &
Distance\\
\includegraphics[height=.22\textheight, clip=true, trim=1cm 1cm 0cm 1.6cm]{draft_figures/descriptive_plots/mass.pdf}    &
\includegraphics[height=.22\textheight, clip=true, trim=1cm 1cm 0cm 1.6cm]{draft_figures/descriptive_plots/distance.pdf}   \\
GDP per capita &
Trade as \% of GDP\\
\includegraphics[height=.22\textheight, clip=true, trim=1cm 1cm 0cm 1.6cm]{draft_figures/descriptive_plots/GDPpc.pdf}    &
\includegraphics[height=.22\textheight, clip=true, trim=.95cm 1cm 0cm 1.6cm]{draft_figures/descriptive_plots/TradeO.pdf}   \\
Trade Volume &
FDI Stock\\
\includegraphics[height=.22\textheight, clip=true, trim=1cm 1cm 0cm 1.6cm]{draft_figures/descriptive_plots/TradeV.pdf}    &
\includegraphics[height=.22\textheight, clip=true, trim=1cm 1cm 0cm 1.6cm]{draft_figures/descriptive_plots/fdi_stock.pdf}   \\
\end{tabular}
\caption{\label{fig:effectPlots4} Summary Statistics.}
\end{figure}

In the gravity model, we include the log product of the dyad's real GDP\footnote{The data comes from the \textit{Penn World Table}  \citep{feenstra2015next}.} and logged Euclidean distance.\footnote{See \citet{mayer2011notes} for the calculation of Euclidean distance.} Generally, larger products of GDP are associated with higher levels of FDI while longer distances are associated with less FDI. One key point here is that for the purpose of model convergence the logged product of dyadic GDP has been estimated as one variable in the model, rather than being estimated separately.

We also control for a country's trade openness (trade as \% of GDP) and GDP growth rate. Existing research has shown that FDI and trade are compliments \citep{aizenman2006fdi,Markusen:1995}. We expect that higher levels of trade openness will be associated with higher levels of FDI. High GDP growth rates stand in for the general health of a country's economy. Thus we expect that a high GDP growth rate to correlate with more FDI, both as a sender and receiver. Both data are from the World Bank's \textit{World Development Indicators}.

There is a substantial amount of work that explores the relationship between democratic institutions and FDI inflows; yet empirical results to date remain inconclusive \citep[see e.g.][]{Henisz:2000,Jensen:2003,Li_Resnick:2003,Jakobsen_DeSoysa:2006,Resnick:2001,Wright_Zhu:2017}. We include standard polity scores as a measure of a country's level of democracy \citep{Marshall_Jaggers:2010}. We also include political violence to proxy for state instability,\footnote{Data comes from \citet{marshall2005major}.} which should be negatively correlated with FDI inflows.

In addition, we include two sets of international agreement variables. The first is four dummy variables for different types of defense agreements, from \citet{Gibler09}. They include defense, entente, non-aggression, and neutrality treaties. We expect these variables to be positively associated with FDI inflows, particularly defense treaties since this indicates political cooperation and low political risk \citep{Li_Vashchilko:2010}. The second is preferential trade agreement (PTAs). Signing a PTA represents a commitment to liberal markets that investors would favor and therefore would be associated with increased FDI inflows \citep{Buthe_Milner:2008,buthe2014foreign}. Yet, PTAs vary significantly in depth with some requiring nearly full liberalization of trade barriers while others are superficial political signals. We thus measure the depth of PTAs by using latent trait analysis with 48 different dichotomous variables regarding topics covered in PTAs.\footnote{Data are from \citet{dur2014design}.}





\subsection{Model and Specification: The Count ERGM}

To model the FDI network, we must use a statistical modeling approach that is capable of representing the dependencies underlying the ties. The literature offers a number of options. These include the latent space family of models, such as those that have been used to model trade networks in political science \citep{ward2007persistent,ward2013gravity}; the generalized exponential random graph model (GERGM), which can be used to model complex network features in networks with continuous-valued edges \citep{desmarais2012statistical,wilson2017stochastic}; and the ERGM for count-valued edges \citep{krivitsky2012exponential}. We select the count-valued ERGM for two reasons. First, if the researcher's objective is to test hypotheses regarding dependent network structure, ERGM family models can accomplish this more precisely than can latent space models \citep{cranmer2016navigating,cranmer2016critique,desmarais2017statistical}. Second, the count ERGM offers a modeling advantage over the GERGM for data such as FDI flows, which are zero for the majority of dyads. That is, the count ERGM is capable of modeling zero inflation in the network. This paper presents, as far as we are aware, the first application in political science of the count ERGM proposed by \cite{krivitsky2012exponential}.

Like other forms of the ERGM, the count ERGM is a statistical model that operates on one or more network adjacency matrices. To specify the count ERGM, the researcher selects two types of network statistics---those that relate tie values to observed covariates (i.e., covariate effects), and those that relate the ties to each other via high order network structure (i.e., network effects). If an ERGM is specified without network effects, it reduces to a dyadic regression model in which ties are assumed to be independent and identically distributed \cite{cranmer2011inferential}. Under \citeapos{krivitsky2012exponential} count ERGM, the probability of the observed $n \times n$ network adjacency matrix $\bm{y}$ is $$ \text{Pr}_{\bm{\theta};h;\bm{g}}( \bm{Y}=\bm{y} )=\frac{ h(\bm{y})\text{exp}( \bm{\theta} \cdot \bm{g} (\bm{y}) )}{\bm{\kappa}_{h,\bm{g}}(\bm{\theta})},$$ where $\bm{g}( \bm{y} )$ is the vector of network statistics used to specify the model, $bm{\theta}$ is the vector of parameters that describes how those statistic values relate to the probability of observing the network, $h(\bm{y})$ is a reference function defined on the support of $\bm{y}$ and selected to affect the shape of the baseline distribution of dyadic data (e.g., Poisson reference measure), and $\bm{\kappa}_{h,\bm{g}}(\bm{\theta})$ is the normalizing constant that assures that the probabilities over all possible networks sums to one.


\subsubsection{Specification}


The count ERGM is extremely flexible in that there are very few constraints on the generative features that can be incorporated into the model through $\bm{g}( \bm{y} )$. In the models we specify, we use statistics that model the shape of the individual edge distributions (i.e., the shapes of directed dyadic FDI flows), model the dependencies we have described above, and account for the effects of exogenous covariates. The statistics we use to account for the individual edge distribution include, $$\text{Sum}:\bm{g(y)} = \sum_{(i,j) {\in} \mathbb{Y}}\bm{y}_{i,j},$$ which models the average edge value $$\text{Sum, Fractional Moment}:\bm{g(y)} = \sum_{(i,j) {\in} \mathbb{Y}}\bm{y}_{i,j}^{1/2},$$ which accounts for dispersion in the edge distribution, and
$$\text{Non-Zero}: \bm{g}_k = \sum_{(i,j) {\in} \mathbb{Y}} \mathbb{I}(\bm{y}_{i,j} \neq 0),$$ which models the prevalence of zeros in dyadic FDI flows. We include two statistics to model the dependencies that correspond to our hypotheses. First,
$$ \text{Reciprocity}: \bm{g(y)} = \sum_{(i,j) {\in} \mathbb{Y}}min(\bm{y}_{i,j},\bm{y}_{j,i}),$$ in which we add up the lowest edge value within each dyad. If edges are reciprocated, this statistic will increase due to the co-occurrence of large edge values within the same dyad. Second,
$$\text{Transitive Weights}: \bm{g(y)} =  \sum_{(i,j) {\in} \mathbb{Y}}\min\bigg( \bm{y}_{i,j}, \max\limits_{k{\in}N}\Big(\min(\bm{y}_{i,k},\bm{y}_{k,j})\Big) \bigg),$$ which acounts for the degree to which edge $(i,j)$ co-occurs with pairs of large edge values with which  edge $(i,j)$ forms a transitive (i.e., non-cyclical) triangle. Exogenous covariates are accounted for with statistics that measure the degree to which large covariate values co-occur with large edge values. First,
$$ \text{Dyadic Covariate}: \bm{g(y,x)} = \sum_{(i,j)} \bm{y}_{i,j}x_{i,j},$$ measures this co-occurrence at the level of the directed dyad, in which there is a dyadic observation of the covariate corresponding to each potential FDI flow. There are two statistics that account for node (i.e., country) level covariates. Each statistic takes the product of the node's covariate value and a sum of the edge values in which the node is involved. The first, ``Sender Covariate,'' uses the sum over the flows that the node sends. The second, ``Receiver Covariate,'' uses the sum over the flows that the node receives. These two variants of node-level statistics differentiate between the effects of a variable on the volume of FDI originating from a state, and being invested in a state, respectively.

$$ \text{Sender Covariate}: \bm{g(y,x)} = \sum_{i}x_i \sum_{j} \bm{y}_{i,j}$$

$$ \text{Receiver Covariate}: \bm{g(y,x)} = \sum_{j}x_j \sum_{i} \bm{y}_{i,j}$$

\noindent The count ERGM estimates that we present below are estimated using the \texttt{ergm} \citep{ergm} and \texttt{ergm.count} \citep{ergmcount} packages in the \R \space statistical software \citep{r}. We estimate a separate model for each year from 2002 to 2012. We have enough data to identify a separate set of parameter values in each year, as we observe over fifteen-thousand potential directed dyadic ties in a single year.  By allowing the parameters to change with each year, we can observe the temporal robustness of effects, and avoid imposing the limiting assumption that the coefficient values are stable.




\begin{figure}[htp]
\centering
\begin{tabular}{c@{\hskip -.4cm}c}
Sum of Edges&
Sum Sqrt. of Edges\\
\includegraphics[height=.22\textheight, clip=true, trim=0cm .5cm 0cm .1cm]{draft_figures/rl_plots/Sum.pdf}    &
\includegraphics[height=.22\textheight, clip=true, trim=.5cm .5cm 0cm .1cm]{draft_figures/rl_plots/Sum_5.pdf}   \\
Number of Non-Zero Edges &
Lagged FDI Flow\\
\includegraphics[height=.22\textheight, clip=true, trim=0cm .5cm 0cm .1cm]{draft_figures/rl_plots/Nonzero.pdf} &
\includegraphics[height=.22\textheight, clip=true, trim=.5cm .5cm 0cm .1cm]{draft_figures/rl_plots/LDV.pdf}   \\
Log-GDP Product &
Log-Geographic Distance\\
\includegraphics[height=.22\textheight, clip=true, trim=0cm .5cm 0cm .1cm]{draft_figures/rl_plots/GDP_Dyad.pdf}    &
\includegraphics[height=.22\textheight, clip=true, trim=.5cm .5cm 0cm .1cm]{draft_figures/rl_plots/Distance.pdf}   \\
Defense Treaty &
Alliance Treaty\\
\includegraphics[height=.22\textheight, clip=true, trim=0cm .5cm 0cm .1cm]{draft_figures/rl_plots/Defense.pdf}   &
\includegraphics[height=.22\textheight, clip=true, trim=.5cm .5cm 0cm .1cm]{draft_figures/rl_plots/Alliance.pdf}   \\
\end{tabular}
\caption{\label{fig:effectPlots1} Estimates of exogenous terms in Poisson ERGMs. Bars span 95\% confidence intervals. Black coefficient representations are from models excluding dependence terms (i.e., transitivity and reciprocity).}
\end{figure}




\begin{figure}[htp]
\centering
\begin{tabular}{c@{\hskip -.4cm}c}
Polity, in-degree &
Polity, out-degree\\
\includegraphics[height=.22\textheight, clip=true, trim=0cm .5cm 0cm .1cm]{draft_figures/rl_plots/Polity_in.pdf}   &
\includegraphics[height=.22\textheight, clip=true, trim=.5cm .5cm 0cm .1cm]{draft_figures/rl_plots/Polity_out.pdf}   \\
GDP per capita, in-degree &
GDP per capita,  out-degree\\
\includegraphics[height=.22\textheight, clip=true, trim=0cm .5cm 0cm .1cm]{draft_figures/rl_plots/GDPpc_in.pdf} &
\includegraphics[height=.22\textheight, clip=true, trim=.5cm .5cm 0cm .1cm]{draft_figures/rl_plots/GDPpc_out.pdf}   \\
Bilateral Investment Treaty &
Bilateral Trade Volume\\
\includegraphics[height=.22\textheight, clip=true, trim=0cm .5cm 0cm .1cm]{draft_figures/rl_plots/BIT.pdf}    &
\includegraphics[height=.22\textheight, clip=true, trim=.5cm .5cm 0cm .1cm]{draft_figures/rl_plots/TradeV.pdf}   \\
Trade Openness, in-degree &
Trade Openness,  out-degree\\
\includegraphics[height=.22\textheight, clip=true, trim=0cm .5cm 0cm .1cm]{draft_figures/rl_plots/TradeO_in.pdf}  &
\includegraphics[height=.22\textheight, clip=true, trim=.5cm .5cm 0cm .1cm]{draft_figures/rl_plots/TradeO_out.pdf}   \\
\end{tabular}
\caption{\label{fig:effectPlots2} Estimates of exogenous terms in Poisson ERGMs. Bars span 95\% confidence intervals. Black coefficient representations are from models excluding dependence terms (i.e., transitivity and reciprocity).}
\end{figure}



\begin{figure}[htp]
\centering
\begin{tabular}{c@{\hskip -.4cm}c}
Reciprocity &
Transitivity\\
\includegraphics[height=.22\textheight, clip=true, trim=0cm .5cm 0cm .1cm]{draft_figures/rl_plots/Mutuality.pdf}    &
\includegraphics[height=.22\textheight, clip=true, trim=.5cm .5cm 0cm .1cm]{draft_figures/rl_plots/Transitivity.pdf}   \\
\end{tabular}
\caption{\label{fig:effectPlots4} Estimates of Network terms in Poisson ERGMs. Bars span 95\% confidence intervals.}
\end{figure}



\section{Results}

The coefficients estimated in the yearly count ERGMs are depicted in Figures \ref{fig:effectPlots1}--\ref{fig:effectPlots4}. Before discussing individual effects, we first assess the relative fit of the independence and network models. Figure \ref{fig:bic} presents the difference in Bayesian Information Criterion (BIC) in the between the independence and network models for each year in our analysis. The BIC is more conservative in terms of adding parameters to a model than the common alternative likelihood-based measure of model fit, the Akaike Information Criterion (AIC) \citep{waldorp2005model,abrahamowicz1990optimal,raftery1999bayes}. We see that the BIC in the independence model is higher than that in the network model for each year, which provides robust evidence that the network model provides a better fit to the data than the independence model over the time period that we study. Turning now to the network effects, which are presented in Figure \ref{fig:effectPlots4}, we see that the reciprocity and transitivity effects are positive and statistically significant in each year, offering robust evidence that FDI flows are interdependent according to these two canonical forms of network structure.'

The dependence effects, though formulated intuitively, do not permit a straightforward marginal-effects interpretation of the coefficients aside from the signs of the effects. We can, however, estimate and visualize the dependence effects using simulation. In Figure \ref{fig:interpret} we present visualizations of the effects of the dependence terms. To measure these effects we begin with a simulation exercise in which we simulate networks using both the full model with dependence terms, and the null model based only on covariates. We then classify each simulated edge value in terms of the value of the value of the local version of the dependence term operating on that edge. For example, when it comes to the reciprocity effect, we classify each simulated edge value ($y_{i,j}$) in terms of the value of the mutual edge, $y_{j,i}$. Finally, we estimate the difference in means between the edge values simulated from the full and null models at each dependence term value. This difference in means can be interpreted as the effect on predicted edge values of accounting for the respective dependence term in the model.  We see in Figure \ref{fig:interpret}, that the dependence effects can result in differences in predicted edge values in the range of 1--4 in log-scale FDI.  The standard deviation in log-scale FDI stock (in 2012---the year we use for the interpretation plots) is 2.40.  We see that the scale of both the reciprocity and transitivity effects are significant, with a shift from lower values of the relevant dependence edge to higher values resulting in more than a standard deviation increase in the predicted edge value.

\begin{figure}[htp]
\centering
\includegraphics[scale=.75]{draft_figures/transitiveInterpretation.pdf} \vspace{-.5cm}\\
\includegraphics[scale=.75]{draft_figures/mutualInterpretation.pdf} \vspace{-.5cm}
\caption{\label{fig:interpret} Plots depict the difference in predicted value ($y$-axis) that is attributable to the respective dependence effect, averaged over all dyads in the network. Interpretation plots are based on 1,000 FDI stock networks simulated from the 2012 model. Tie weights are measured on the natural logarithm scale. Predicted value differences are calculated by taking the differences between expected dyad values simulated from the full model with dependence terms and the null model that is based on covariates only. Error bars span 95\% confidence intervals for the difference in means. }
\end{figure}



We noted above that omitting dependent network structure, a condition that characterizes previous research on FDI, can result in biased estimates and improper standard errors. For several effects that we include in our models, the results are substantively changed by adding the network parameters. In the network model, we find the following effects to be lower in magnitude, statistically significant in fewer years, or both: Gravity model mass, distance, contiguity, PTA depth, destination polity, destination trade openness, origin trade openness, origin GDP per capita, origin polity, and origin trade openness. For each of these effects, our results indicate that omitting the network dependencies lead to either an overestimate of the effect of the respective variable, or worse, a Type 2 inferential error in which the null hypothesis of no effect is incorrectly rejected. This finding shows that, even if a researcher is not theoretically interested in network dependencies, (s)he should still incorporate them into an empirical model in order to avoid misspecification bias.

\begin{figure}[htp]
\centering
\includegraphics[scale=.75]{draft_figures/BICdiff.pdf} \vspace{-.5cm}
\caption{\label{fig:bic} Difference in BIC between independent and network model.}
\end{figure}

\subsection{Pooled ERGM Results}

\begin{figure}[htp]
\centering
\begin{tabular}{c@{\hskip 0cm}c}
Mutuality & Transitivity \\
\includegraphics[height=.22\textheight, clip=true, trim=0cm 0cm 0cm .2cm]{draft_figures/plots_pooled/Mutuality.pdf}    &
\includegraphics[height=.22\textheight, clip=true, trim=0cm 0cm 0cm .2cm]{draft_figures/plots_pooled/Transitivity.pdf}
\end{tabular}
\caption{\label{fig:effectPlots4} Estimates of Dependence terms in time-pooled ERGMs. Bars span 95\% confidence intervals. }
\end{figure}
 


\begin{figure}[htp]
\centering
\begin{tabular}{c@{\hskip 0cm}c}
Sum & Sum$^{(1/2)}$ \\
\includegraphics[height=.22\textheight, clip=true, trim=0cm 0cm 0cm .2cm]{draft_figures/plots_pooled/Sum.pdf}    &
\includegraphics[height=.22\textheight, clip=true, trim=0cm 0cm 0cm .2cm]{draft_figures/plots_pooled/sum_5.pdf}   \\
Non-Zero & Alliance Treaty\\
\includegraphics[height=.22\textheight, clip=true, trim=0cm 0cm 0cm .2cm]{draft_figures/plots_pooled/Non-zero.pdf} &
\includegraphics[height=.22\textheight, clip=true, trim=0cm 0cm 0cm .2cm]{draft_figures/plots_pooled/AllianceTreaty.pdf}   \\
Bilateral Investment Treaty & Defense Treaty\\
\includegraphics[height=.22\textheight, clip=true, trim=0cm 0cm 0cm .2cm]{draft_figures/plots_pooled/BIT.pdf} &
\includegraphics[height=.22\textheight, clip=true, trim=0cm 0cm 0cm .2cm]{draft_figures/plots_pooled/DefenseTreaty.pdf}   \\
Product of Dyad's GDP & Distance\\
\includegraphics[height=.22\textheight, clip=true, trim=0cm 0cm 0cm .2cm]{draft_figures/plots_pooled/DyadGDPProduct.pdf} &
\includegraphics[height=.22\textheight, clip=true, trim=0cm 0cm 0cm .2cm]{draft_figures/plots_pooled/Distance.pdf}   \\
\end{tabular}
\caption{\label{fig:effectPlots4} Estimates of terms in time-pooled ERGMs. Bars span 95\% confidence intervals. Black coefficient representations are from models excluding dependence terms (i.e., transitivity and reciprocity).}
\end{figure}

\begin{figure}[htp]
\centering
\begin{tabular}{c@{\hskip 0cm}c}
Lagged FDI stock & Bilateral Trade Volume \\
\includegraphics[height=.22\textheight, clip=true, trim=0cm 0cm 0cm .2cm]{draft_figures/plots_pooled/LaggedDV.pdf}    &
\includegraphics[height=.22\textheight, clip=true, trim=0cm 0cm 0cm .2cm]{draft_figures/plots_pooled/TradeVolume.pdf}   \\
GDP per capita, in-degree & GDP per capita, out-degree\\
\includegraphics[height=.22\textheight, clip=true, trim=0cm 0cm 0cm .2cm]{draft_figures/plots_pooled/GDPpc_in.pdf} &
\includegraphics[height=.22\textheight, clip=true, trim=0cm 0cm 0cm .2cm]{draft_figures/plots_pooled/GDPpc_out.pdf}   \\
Polity, in-degree & Polity, out-degree\\
\includegraphics[height=.22\textheight, clip=true, trim=0cm 0cm 0cm .2cm]{draft_figures/plots_pooled/Polity_in.pdf} &
\includegraphics[height=.22\textheight, clip=true, trim=0cm 0cm 0cm .2cm]{draft_figures/plots_pooled/Polity_out.pdf}   \\
Trade Openness, in-degree & Trade Openness, out-degree\\
\includegraphics[height=.22\textheight, clip=true, trim=0cm 0cm 0cm .2cm]{draft_figures/plots_pooled/Trade_in.pdf} &
\includegraphics[height=.22\textheight, clip=true, trim=0cm 0cm 0cm .2cm]{draft_figures/plots_pooled/Trade_out.pdf}   \\
\end{tabular}
\caption{\label{fig:effectPlots4} Estimates of terms in time-pooled ERGMs. Bars span 95\% confidence intervals. Black coefficient representations are from models excluding dependence terms (i.e., transitivity and reciprocity).}
\end{figure}



\newpage
\section{Conclusion}
% paragraph on substantive finding

In the past decades, one prominent feature of the global economy is the growth of global production networks. Firms have chosen to invest overseas at an unprecedent level, and consequently, production is increasingly fragmented and organized across the globe. One central question is then what drives the pattern of global investment flows. In this paper, we adopt a novel network approach to address this question. FDI flows represent ties between states that arise through both a complex underlying network of inter and intra-firm relations, and legal agreements between states. The relational backdrop through which FDI operates leads to predictable network structure in the patterns of ties formed through FDI. We present a network theory of FDI that includes reciprocity and transitivity as the core structural dependencies. The results of our statistical models confirm that these dependencies exist---a result that holds over time, and while adjusting for other covariates known to relate to FDI. Our result bears important real-world implications, as network dependencies will lead to the effects of policies relevant to FDI to ripple through the network according to these dependencies.

Our paper makes a few important contributions to the literature. First, to our knowledge, we are the first to examine FDI flows through a network approach. Our network theory suggests that FDI flows can be shaped by its network structures. This has been overlooked in the existing literature that focuses exclusively on countries or dyads' political and economic characteristics.\footnote{See \citet{Pandya:2016} for a review of the literature.} We believe our network perspective to global investment flows have broad implications for other cross-border movements of aid, goods, services, people, etc., which are central themes in the IPE literature. Global international trade regimes, for instance, are explicitly designed based on the principle of reciprocity \citep{Bagwell_Staiger:1999}. Yet empirical studies of trade flows rarely account for the pattern of reciprocity. Likewise, we expect other global economic exchanges to exhibit structural characteristics as well.

% paragraph on methodological contribution
Second, we offer a methodological contribution to the literature on FDI, and political science more broadly. In regards to the study of FDI,  we demonstrate how the count ERGM can be used to model the effects of both covariates and network dependencies on FDI flows. We show that adding network dependencies to the covariate-based model of FDI offers a robust improvement in model fit. Yet, ignoring network dependencies can lead to biased estimates or even invalid inferences. In future work on FDI, researchers should consider using the count ERGM, or comparable models for weighted networks.

Third, we introduced the count ERGM in political science. This model is broadly applicable to weighted network data, and, as we demonstrate, offers the powerful capability to represent precise network theory along side covariate effects, handles zero inflation, and can be used for either single network analysis or pooled over multiple networks. [Note: can we say this model has a wider application in the IPE field, and political science in general, because it can handle continuous DVs with network dependencies?]

Finally, we should emphasize that our theory, specification, and finding of network-wide reciprocity and transitivity represent just the start in a broader scholarly dialogue on the network science of FDI flows. One limitation of our study is that we do not model the conditional reciprocity and transitivity. In theory, we should expect that the degree of reciprocity varies by countries' levels of development. Investing abroad incurs large fixed costs and firms need to overcome the disadvantages such as ``liability of foreignness'' they face when competing with indigenous firms in the host country. Therefore, only the most productive firms are able to engage in FDI activities \citep{Melitz:2003,Helpman_et_al:2004}. Historically, MNCs from developed countries predominate. Although there is a surge of FDI from developing countries since the early 2000s, firms in most developing countries are still not competitive enough to strive in a global market.\footnote{For instance, in 2005 outward FDI flows and stocks from developing countries are approximately 17\% and 13\% of the world total, respectively \citep{UNCTAD:2006}. Furthermore, outward FDI from developing countries is highly concentrated; the top 10 countries, mostly large emerging economies such as Argentina, Brazil, Chile, China, Mexico, Russia, and South Africa contribute about 83\% \citep{UNCTAD:2006}. } Future research should explore how network dependencies vary across different groups of countries.




\newpage
\singlespacing
\bibliographystyle{apsr}
\bibliography{fdi_reference}




\end{document}



Table \ref{tab:describe_binary} provides means for the dichotomous dyadic variables used in our models....

% Table created by stargazer v.5.2 by Marek Hlavac, Harvard University. E-mail: hlavac at fas.harvard.edu
% Date and time: Mon, Feb 20, 2017 - 22:08:59
\begin{table}[htp] \centering
  \caption{}
  \label{}
\begin{tabular}{@{\extracolsep{5pt}}lcccc}
\\[-1.8ex]\hline
\hline \\[-1.8ex]
Statistic &  \multicolumn{1}{c}{Mean} & \multicolumn{1}{c}{St. Dev.} & \multicolumn{1}{c}{Min} & \multicolumn{1}{c}{Max} \\
\hline \\[-1.8ex]
Contiguity &  0.024 & 0.152 & 0 & 1 \\
Common Official Language &  0.112 & 0.315 & 0 & 1 \\
Common Language and Ethnicity &  0.115 & 0.318 & 0 & 1 \\
Former Colonial Relationship &  0.015 & 0.121 & 0 & 1 \\
Common Colonizer &  0.062 & 0.241 & 0 & 1 \\
Defense Treaty &  0.075 & 0.264 & 0 & 1 \\
Non-aggression Treaty &  0.064 & 0.245 & 0 & 1 \\
Neutrality Treaty &  0.004 & 0.063 & 0 & 1 \\
Entente Treaty &  0.066 & 0.248 & 0 & 1 \\
\hline \\[-1.8ex]
\end{tabular}
\caption{\label{tab:describe_binary} Descriptive statistics for dichotomous dyadic covariates. Number of observations across all years is 189,000.}
\end{table}


\includegraphics[scale=.8]{draft_figures/reciprocity.png}\\
\includegraphics[scale=.8]{draft_figures/transitivity.png}\\
\includegraphics[scale=.8]{draft_figures/assortativity.png}\\
