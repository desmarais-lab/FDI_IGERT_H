%\title{Overleaf Memo Template}
% Using the texMemo package by Rob Oakes
\documentclass[a4paper,11pt]{texMemo}
\usepackage[english]{babel}
\usepackage{graphicx, lipsum}

%% Edit the header section here. To include your
%% own logo, upload a file via the files menu.
\memosubject{Manuscript Revisions for \emph{Political Science Research and Methods}}
\memoto{PSRM-OA-2019-0085.R1
}
\memofrom{Complex Dependence in Foreign Direct Investment: Network Theory and Empirical Analysis}
\memodate{\today}


\begin{document}
\maketitle

\noindent We thank the editors for the opportunity to publish our manuscript in PSRM. In this memo
we briefly document the edits we have made in response to the editor's comments in the conditional acceptance letter. We agree with the feedback provided, and believe that the incorporation of additional discussion regarding the differences between spatial and network approaches has improved the paper.


\section*{Editor}

\noindent \textbf{E1:} \emph{ Both reviewers think your manuscript should be published, but R1 remains somewhat skeptical that network models are the best way to analyze FDI. I agree that the manuscript is stronger if you can make the case that ERGMs offer something that spatial models cannot. You note that the spatial literature is mostly monadic. This is a good point (although the N/P model is dyadic). Interestingly, R1 cites a paper by Metulini et al. in the review below that may be helpful for a couple of reasons: (1) they use a count model for trade flows and (2) their approach is to filter out spatial dependence. If the available spatial models that address zero inflation use count models, but treat the (spatial) interdependence as a nuisance, then I think your approach, in which the interdependence is central to the analysis, is clearly superior. If you agree, you might add something brief along these lines.  }\\

\noindent \textbf{Addressed:} We followed this suggestion exactly. In the paragraph starting at the end of Page 5, we cited the Metulini et al. paper as providing a model that could be used to adjust for dependence in dyadic FDI, and would be particularly appropriate because it is capable of adjusting for zero inflation. We then noted, however, that the modeling approach they propose does not permit the direct estimation of dependence parameters, drawing a contrast with the network approach we propose.  \\


\end{document} 