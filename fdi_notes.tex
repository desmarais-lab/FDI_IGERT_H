\documentclass{article}
\usepackage[margin=1in]{geometry}
\usepackage{longtable}
\usepackage{hyperref}
\hypersetup{
    colorlinks=true,
    linkcolor=red,
    filecolor=magenta,      
    urlcolor=red,
}
\usepackage{graphicx}
\usepackage{setspace}
\usepackage[table]{xcolor}
\usepackage{booktabs}
\usepackage{courier}
\begin{document}

\title{FDI Project Notes}
\author{IGERT}
\date{Fall 2016}
\maketitle

\singlespacing
\begin{abstract} 
    \noindent Our paper examines the determinants of foreign direct investment (FDI) inflows within a network structure. The political economy of FDI literature has shown that a number of political and economic variables such as security alliances, preferential trade agreements, migration networks, and colonial history shape the patterns of FDI flows. However, most existing studies based on monadic or dyadic models overlook the complex dependencies in the network. Global FDI flows operate as a network and therefore the independence assumptions of generalized linear models are not met and network dependencies such as reciprocity and transitivity are not controlled for. In this paper, we utilize bilateral FDI inflow data from UNCTAD and examine the political and economic networks of FDI flows using generalized exponential random graph models (GERGM). These models allow researchers to control for higher order dependencies as well as node and edge variables within the same model. The GERGM uses Markov Chain Monte Carlo algorithms to simulate weighted graphs from which the likelihood of coefficients can be estimated.

\end{abstract}

\section{Introduction}

Research examining foreign direct investment (FDI) and its relationship with economic and political determinants is expansive. Much of this work is conducted using the gravity model, which was originally developed to predict trade flows. This framework models FDI flows using dyadic data and the product of partner GDPs as mass and some variant of distance as an independent variable. Our work highlights a key weakness of these models that rely on standard panel regression models. There has been a growing body of literature that brings into question the way we estimate models for dyadic data (add list of papers). The primary challenge is that dyadic data is an edge-list and therefore represents a network. Ignoring this unmodeled network structure violates assumptions within a generalized linear model, potentially leading to biased estimates.
\newpage


\section{Preliminary Valued ERGM Results}

\begin{enumerate}
\item Model 1: Only fit with network terms
\item Model 2: Add gravity variables
	
\end{enumerate}



\begin{table}[!htb]
\begin{center}
\begin{tabular}{l c c }
\hline
 & Model 1 & Model 2 \\
\hline
sum                           & $2.83^{***}$  & $-0.14^{***}$ \\
                              & $(0.04)$      & $(0.01)$      \\
sum(1/2)                        & $-7.41^{***}$ & $-3.44^{***}$ \\
                              & $(0.15)$      & $(0.02)$      \\
nonzero                       & $2.25^{***}$  & $-0.13^{***}$ \\
                              & $(0.13)$      & $(0.02)$      \\
mutual, geometric              & $0.69^{***}$  & $0.27^{***}$  \\
                              & $(0.02)$      & $(0.03)$      \\
GDP product, logged     &               & $0.01^{***}$  \\
                              &               & $(0.00)$      \\
Distance, logged &               & $-0.19^{***}$ \\
                              &               & $(0.01)$      \\
\hline
AIC                           & -16240.55     & -21673.28     \\
BIC                           & -16209.90     & -21627.29     \\
Log Likelihood                & 8124.28       & 10842.64      \\
\hline
\multicolumn{3}{l}{\scriptsize{$^{***}p<0.001$, $^{**}p<0.01$, $^*p<0.05$}}
\end{tabular}
\caption{Statistical models}
\label{table:coefficients}
\end{center}
\end{table}
\doublespacing

\newpage

\section{Outline}
\begin{itemize}
  \item{Intro}
    \begin{itemize}
      \item{Paragraph about the background on networks in polisci}
     	 \begin{itemize}
      	\item{Recent discussion in ISQ (Vol 2, issue 60) and other work}
      	\item{Mention that there is even less work on FDI network}
      	\end{itemize}
	\item{Outline of Paper}
    \end{itemize}
  \item{Methods background}
   \begin{itemize}
      	\item{Discussion of the possible approaches OLS with FE, Latent, ERGM}
      	\item{Explanation of what is ERGM and why it is preferable}
      	\end{itemize}
  \item{Discussion of theoretical expectations}
  \begin{itemize}
      	\item{Explanation of substantive reasons for network terms}
      	\item{Explanation of covariates}
      	\end{itemize}
   \item{Discussion of data sources and model}
     \item{Results}
     \item{Conclusion and future work}
\end{itemize}





\newpage
\section{Data Collection}



\begin{enumerate}
	\item {FDI Data}
		\begin{itemize}
			\item{Downloaded from UNCTAD}
			\item{Cleaned and appended into panel format}
		\end{itemize}
	\item {GDP}
		\begin{itemize}
			\item{Downloaded from WB WDI}
		\end{itemize}
	\item {Distance}
		\begin{itemize}
			\item{Downloaded from CEPII}
			\item{Mayer, T. \& Zignago, S. (2011) Notes on CEPII?s distances measures: the GeoDist Database
					CEPII Working Paper 2011-25}
			\item{Includes Common Language, Contiguity, Ethnolinguistic similarity, and Colonial ties}
		\end{itemize}
	\item {Alliances}
		\begin{itemize}
			\item{Downloaded from COW}
			\item{Gibler, Douglas M. 2009. International military alliances, 1648-2008. CQ Press}
		\end{itemize}
	\item {Regime Type}
		\begin{itemize}
			\item{Downloaded from Center for Systemic Peace}
			\item{http://www.systemicpeace.org/inscrdata.html}
		\end{itemize}
	\item {GDP growth rate}
		\begin{itemize}
			\item{Downloaded from WB WDI}
		\end{itemize}
	\item {Trade Openness}
		\begin{itemize}
			\item{Downloaded from WB WDI}
		\end{itemize}
	\item {Population}
		\begin{itemize}
			\item{Downloaded from WB WDI}
		\end{itemize}
	\item {Transparency}
		\begin{itemize}
			\item{Downloaded from Transparency International}
			\item{Not included in preliminary results, subsets countries to 75}
		\end{itemize}
	\item {PTA Depth}
		\begin{itemize}
			\item{Downloaded from DESTA}
			\item{http://www.designoftradeagreements.org}
		\end{itemize}
	\item {Trade}
		\begin{itemize}
			\item{Downloaded from OECD.stat}
			\item{http://stats.oecd.org/}
			\item{includes trade in household consumption items, intermediate goods, capital goods, and mixed end-use goods}
		\end{itemize}
		
\end{enumerate}

\section{Data Preparation}
\begin{enumerate}
	\item Downloading Data: \texttt{FDI\_download.R}
		\begin{itemize}
			\item{STP and CIV cannot be scraped due to Uni-code issues:Downloaded and formatted manually}
			\item{Had to rewrite ``Sao Tao and Principe'' in its file; Changed ``Cote d?Voire'' to ``Ivory Coast''}
			\item{Does not run from makefile due to these manual changes}
		\end{itemize}
	\item Cleaning Data: \texttt{FDI\_clean.R}
		\begin{itemize}
			\item{Gets rid of empty columns and empty rows, then appends all countries to together}
			\item{Gets rid of aggregates and adds country codes}
			\item File output \texttt{fdi\_clean.csv}
		\end{itemize}
	\item Adding data: \texttt{FDI\_merge.R}
		\begin{itemize}
			\item{Adds control variables}
			\item File output: replaces \texttt{fdi\_merge.csv} 
		\end{itemize}
	\item Subset data: \texttt{FDI\_subset.R}
		\begin{itemize}
			\item{Subsets data to create full edge list for all variables}
		\end{itemize}
	\item Preliminary Results: \texttt{fdi\_MRQAP.R}
		\begin{itemize}
			\item{Runs OLS models}
			\item Output: FE and MRQAP results 
		\end{itemize}

		
\end{enumerate}

\section{Preliminary Panel Regression Results}

% Table created by stargazer v.5.2 by Marek Hlavac, Harvard University. E-mail: hlavac at fas.harvard.edu
% Date and time: Mon, Dec 12, 2016 - 09:16:44
\begin{longtable}{lr}
  \caption{IM/EX FE effects}\\
  \hline
  IV&Dependent Variable: FDI\\
  \hline
  \endhead
Contiguity & 174.960$^{***}$ \\ 
  & (8.220) \\ 

Common Language\_off & 7.157 \\ 
  & (6.776) \\ 

Common Ethnoliguisitics & $-$7.846 \\ 
  & (6.869) \\ 

Past Colony & 142.077$^{***}$ \\ 
  & (10.464) \\ 

Common Colonizer & 41.110$^{***}$ \\ 
  & (5.583) \\ 

Common Currency & 121.027$^{*}$ \\ 
  & (73.033) \\ 

Distance & $-$0.005$^{***}$ \\ 
  & (0.0004) \\ 

 Destination GDP & 0.000$^{***}$ \\ 
  & (0.000) \\ 

 Origin GDP & 0.000$^{***}$ \\ 
  & (0.000) \\ 

Defense Alliance & 12.256 \\ 
  & (13.213) \\ 

Non-agression Treaty & 154.289$^{***}$ \\ 
  & (18.991) \\ 

Neutrality Treaty & $-$247.209$^{***}$ \\ 
  & (28.128) \\ 

Entente & 131.783$^{***}$ \\ 
  & (20.122) \\ 

 Destination Polity & $-$3.083$^{***}$ \\ 
  & (0.784) \\ 

 Origin Polity & $-$2.265$^{***}$ \\ 
  & (0.785) \\ 

 Destination Trade Opennes & 0.633$^{***}$ \\ 
  & (0.081) \\ 

 Destination GDP growth (\%) & $-$0.682$^{***}$ \\ 
  & (0.245) \\ 

 Origin Trade Opennes & 0.790$^{***}$ \\ 
  & (0.081) \\ 

 Origin GDP growth (\%) & $-$0.876$^{***}$ \\ 
  & (0.245) \\ 

 Origin Population & $-$0.00000$^{***}$ \\ 
  & (0.00000) \\ 

 Destination Political Violence & 0.401 \\ 
  & (2.016) \\ 
 
Origin Political Violence & 6.296$^{***}$ \\ 
  & (2.016) \\ 

PTA depth & 2.615 \\ 
  & (1.692) \\ 

Trade in Household Consumption Goods & $-$0.00004$^{***}$ \\ 
  & (0.00000) \\ 

Trade in Intermediate Goods & 0.0001$^{***}$ \\ 
  & (0.00000) \\ 

Trade in Intermediate Goods Sq. & $-$0.000$^{***}$ \\ 
  & (0.000) \\ 

Trade in Capital Goods & 0.00001$^{***}$ \\ 
  & (0.00000) \\ 

Trade in Mixed End-use Goods & 0.0001$^{***}$ \\ 
  & (0.00000) \\ 

 Constant & $-$136.288$^{***}$ \\ 
  & (21.681) \\ 

\hline \\[-1.8ex] 
Observations & 189,000 (126 Countries, 12 years)\\ 
R$^{2}$ & 0.317 \\ 
Adjusted R$^{2}$ & 0.316 \\ 
Residual Std. Error & 494.613 (df = 188710) \\ 
F Statistic & 303.323$^{***}$ (df = 289; 188710) \\ 
\end{longtable} 

\section{Model changes to make}
\begin{enumerate}
\item 
\item Compare panel vs. regression models vs. latent-space
\item Model security-alliance variables
\item Follow Polity explanations like Mansfield et. al. 
\item Is there a trade network paper to model?
\item Limit categories in valued ERGM counts
\item Use cubic root of DV for robustness
	
\end{enumerate}

\end{document}
